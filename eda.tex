\PassOptionsToPackage{unicode=true}{hyperref} % options for packages loaded elsewhere
\PassOptionsToPackage{hyphens}{url}
%
\documentclass[12pt,ignorenonframetext,aspectratio=169]{beamer}
\usepackage{pgfpages}
\setbeamertemplate{caption}[numbered]
\setbeamertemplate{caption label separator}{: }
\setbeamercolor{caption name}{fg=normal text.fg}
\beamertemplatenavigationsymbolsempty
% Prevent slide breaks in the middle of a paragraph:
\widowpenalties 1 10000
\raggedbottom
\setbeamertemplate{part page}{
\centering
\begin{beamercolorbox}[sep=16pt,center]{part title}
  \usebeamerfont{part title}\insertpart\par
\end{beamercolorbox}
}
\setbeamertemplate{section page}{
\centering
\begin{beamercolorbox}[sep=12pt,center]{part title}
  \usebeamerfont{section title}\insertsection\par
\end{beamercolorbox}
}
\setbeamertemplate{subsection page}{
\centering
\begin{beamercolorbox}[sep=8pt,center]{part title}
  \usebeamerfont{subsection title}\insertsubsection\par
\end{beamercolorbox}
}
\AtBeginPart{
  \frame{\partpage}
}
\AtBeginSection{
  \ifbibliography
  \else
    \frame{\sectionpage}
  \fi
}
\AtBeginSubsection{
  \frame{\subsectionpage}
}
\usepackage{lmodern}
\usepackage{amssymb,amsmath}
\usepackage{ifxetex,ifluatex}
\usepackage{fixltx2e} % provides \textsubscript
\ifnum 0\ifxetex 1\fi\ifluatex 1\fi=0 % if pdftex
  \usepackage[T1]{fontenc}
  \usepackage[utf8]{inputenc}
  \usepackage{textcomp} % provides euro and other symbols
\else % if luatex or xelatex
  \usepackage{unicode-math}
  \defaultfontfeatures{Ligatures=TeX,Scale=MatchLowercase}
\fi
\usetheme[]{metropolis}
% use upquote if available, for straight quotes in verbatim environments
\IfFileExists{upquote.sty}{\usepackage{upquote}}{}
% use microtype if available
\IfFileExists{microtype.sty}{%
\usepackage[]{microtype}
\UseMicrotypeSet[protrusion]{basicmath} % disable protrusion for tt fonts
}{}
\IfFileExists{parskip.sty}{%
\usepackage{parskip}
}{% else
\setlength{\parindent}{0pt}
\setlength{\parskip}{6pt plus 2pt minus 1pt}
}
\usepackage{hyperref}
\hypersetup{
            pdftitle={Exploratory Data Analysis},
            pdfauthor={Gabriel Lahman; Rowan Lavelle; Jordan Majoros; Matthew Zelenin},
            pdfborder={0 0 0},
            breaklinks=true}
\urlstyle{same}  % don't use monospace font for urls
\newif\ifbibliography
\usepackage{color}
\usepackage{fancyvrb}
\newcommand{\VerbBar}{|}
\newcommand{\VERB}{\Verb[commandchars=\\\{\}]}
\DefineVerbatimEnvironment{Highlighting}{Verbatim}{commandchars=\\\{\}}
% Add ',fontsize=\small' for more characters per line
\usepackage{framed}
\definecolor{shadecolor}{RGB}{248,248,248}
\newenvironment{Shaded}{\begin{snugshade}}{\end{snugshade}}
\newcommand{\AlertTok}[1]{\textcolor[rgb]{0.94,0.16,0.16}{#1}}
\newcommand{\AnnotationTok}[1]{\textcolor[rgb]{0.56,0.35,0.01}{\textbf{\textit{#1}}}}
\newcommand{\AttributeTok}[1]{\textcolor[rgb]{0.77,0.63,0.00}{#1}}
\newcommand{\BaseNTok}[1]{\textcolor[rgb]{0.00,0.00,0.81}{#1}}
\newcommand{\BuiltInTok}[1]{#1}
\newcommand{\CharTok}[1]{\textcolor[rgb]{0.31,0.60,0.02}{#1}}
\newcommand{\CommentTok}[1]{\textcolor[rgb]{0.56,0.35,0.01}{\textit{#1}}}
\newcommand{\CommentVarTok}[1]{\textcolor[rgb]{0.56,0.35,0.01}{\textbf{\textit{#1}}}}
\newcommand{\ConstantTok}[1]{\textcolor[rgb]{0.00,0.00,0.00}{#1}}
\newcommand{\ControlFlowTok}[1]{\textcolor[rgb]{0.13,0.29,0.53}{\textbf{#1}}}
\newcommand{\DataTypeTok}[1]{\textcolor[rgb]{0.13,0.29,0.53}{#1}}
\newcommand{\DecValTok}[1]{\textcolor[rgb]{0.00,0.00,0.81}{#1}}
\newcommand{\DocumentationTok}[1]{\textcolor[rgb]{0.56,0.35,0.01}{\textbf{\textit{#1}}}}
\newcommand{\ErrorTok}[1]{\textcolor[rgb]{0.64,0.00,0.00}{\textbf{#1}}}
\newcommand{\ExtensionTok}[1]{#1}
\newcommand{\FloatTok}[1]{\textcolor[rgb]{0.00,0.00,0.81}{#1}}
\newcommand{\FunctionTok}[1]{\textcolor[rgb]{0.00,0.00,0.00}{#1}}
\newcommand{\ImportTok}[1]{#1}
\newcommand{\InformationTok}[1]{\textcolor[rgb]{0.56,0.35,0.01}{\textbf{\textit{#1}}}}
\newcommand{\KeywordTok}[1]{\textcolor[rgb]{0.13,0.29,0.53}{\textbf{#1}}}
\newcommand{\NormalTok}[1]{#1}
\newcommand{\OperatorTok}[1]{\textcolor[rgb]{0.81,0.36,0.00}{\textbf{#1}}}
\newcommand{\OtherTok}[1]{\textcolor[rgb]{0.56,0.35,0.01}{#1}}
\newcommand{\PreprocessorTok}[1]{\textcolor[rgb]{0.56,0.35,0.01}{\textit{#1}}}
\newcommand{\RegionMarkerTok}[1]{#1}
\newcommand{\SpecialCharTok}[1]{\textcolor[rgb]{0.00,0.00,0.00}{#1}}
\newcommand{\SpecialStringTok}[1]{\textcolor[rgb]{0.31,0.60,0.02}{#1}}
\newcommand{\StringTok}[1]{\textcolor[rgb]{0.31,0.60,0.02}{#1}}
\newcommand{\VariableTok}[1]{\textcolor[rgb]{0.00,0.00,0.00}{#1}}
\newcommand{\VerbatimStringTok}[1]{\textcolor[rgb]{0.31,0.60,0.02}{#1}}
\newcommand{\WarningTok}[1]{\textcolor[rgb]{0.56,0.35,0.01}{\textbf{\textit{#1}}}}
\usepackage{graphicx,grffile}
\makeatletter
\def\maxwidth{\ifdim\Gin@nat@width>\linewidth\linewidth\else\Gin@nat@width\fi}
\def\maxheight{\ifdim\Gin@nat@height>\textheight\textheight\else\Gin@nat@height\fi}
\makeatother
% Scale images if necessary, so that they will not overflow the page
% margins by default, and it is still possible to overwrite the defaults
% using explicit options in \includegraphics[width, height, ...]{}
\setkeys{Gin}{width=\maxwidth,height=\maxheight,keepaspectratio}
\setlength{\emergencystretch}{3em}  % prevent overfull lines
\providecommand{\tightlist}{%
  \setlength{\itemsep}{0pt}\setlength{\parskip}{0pt}}
\setcounter{secnumdepth}{0}

% set default figure placement to htbp
\makeatletter
\def\fps@figure{htbp}
\makeatother


\title{Exploratory Data Analysis}
\providecommand{\subtitle}[1]{}
\subtitle{STAT-S 431}
\author{Gabriel Lahman \and Rowan Lavelle \and Jordan Majoros \and Matthew Zelenin}
\date{Fall 2019}

\begin{document}
\frame{\titlepage}

\begin{frame}

\#\#Introduction

Dr.~Mary Beth Nebel created this study to examine the relationship
between motor skills and social defecits in children with developmental
disorders. The children completed a motor skills assessment as well as
completed a test for intelligence (i.e.~vocabularly, general knowledge,
word similarities, block patterns, and picture concepts).

This experiment seeked to provide an insight as to whether you could
predict a child's social deficit based on their motor abilities.

The developmental disorders that were examined in this study were ADHD
and Autism.

\end{frame}

\begin{frame}{Research Question}
\protect\hypertarget{research-question}{}

Is there a statistically significant difference in predictive power of
the motor skills vs.~social responsivness model when we include
intelligence metrics as predictors?

\end{frame}

\begin{frame}[fragile]{Filtering Data}
\protect\hypertarget{filtering-data}{}

\small

\begin{Shaded}
\begin{Highlighting}[]
\NormalTok{full_dat =}\StringTok{ }\NormalTok{full_dat }\OperatorTok\StringTok{ }\KeywordTok{drop_na}\NormalTok{(}\KeywordTok{c}\NormalTok{(}\StringTok{'SecondaryDiagnosis'}\NormalTok{, }
                                  \StringTok{'CurrentlyNotTakingMeds'}\NormalTok{, }
                                  \StringTok{'CurrentlyTakingAtomoxetine'}\NormalTok{, }
                                  \StringTok{'CurrentlyTakingClonidine'}\NormalTok{, }
                                  \StringTok{'mABC_TotalStandardScore'}\NormalTok{, }
                                  \StringTok{'EdinburghHandedness_Integer'}\NormalTok{))}
\NormalTok{full_dat =}\StringTok{ }\KeywordTok{subset}\NormalTok{(full_dat, visit }\OperatorTok{==}\StringTok{ }\DecValTok{1}\NormalTok{)}
\NormalTok{full_dat =}\StringTok{ }\KeywordTok{subset}\NormalTok{(full_dat, }\OperatorTok{!}\KeywordTok{is.na}\NormalTok{(WISC_VERSION) }\OperatorTok{&}\StringTok{ }\OperatorTok{!}\KeywordTok{is.na}\NormalTok{(SRS_VERSION))}
\NormalTok{full_dat =}\StringTok{ }\KeywordTok{subset}\NormalTok{(full_dat, ADHD_Subtype }\OperatorTok\StringTok{ }\KeywordTok{c}\NormalTok{(}\StringTok{'Combined'}\NormalTok{, }
                                                \StringTok{'Hyperactive/Impulsive'}\NormalTok{, }
                                                \StringTok{'Inattentive'}\NormalTok{, }\StringTok{'No dx'}\NormalTok{))}
\NormalTok{WISC4_full_dat =}\StringTok{ }\KeywordTok{subset}\NormalTok{(full_dat, WISC_VERSION }\OperatorTok{==}\StringTok{ }\DecValTok{4}\NormalTok{)}
\NormalTok{WISC5_full_dat =}\StringTok{ }\KeywordTok{subset}\NormalTok{(full_dat, WISC_VERSION }\OperatorTok{==}\StringTok{ }\DecValTok{5}\NormalTok{)}
\end{Highlighting}
\end{Shaded}

\normalsize

\end{frame}

\begin{frame}[fragile]{Summarizing Data}
\protect\hypertarget{summarizing-data}{}

\tiny

\begin{verbatim}
##         
##          Combined Hyperactive/Impulsive Inattentive No dx
##   ADHD        150                     3          32     0
##   Autism       61                     2          30    39
##   None          0                     0           0   251
\end{verbatim}

\begin{verbatim}
## 
##   4   5 
## 536  32
\end{verbatim}

\begin{verbatim}
##         
##          Combined Hyperactive/Impulsive Inattentive No dx
##   ADHD        150                     3          32     0
##   Autism       55                     2          20    37
##   None          0                     0           0   237
\end{verbatim}

\normalsize

\end{frame}

\begin{frame}{Motor Skills and Social Responsivness Score}
\protect\hypertarget{motor-skills-and-social-responsivness-score}{}

\includegraphics{eda_files/figure-beamer/unnamed-chunk-5-1.pdf}

\end{frame}

\begin{frame}{Log Motor Skills and Social Responsivness Score by
Diagnosis}
\protect\hypertarget{log-motor-skills-and-social-responsivness-score-by-diagnosis}{}

\includegraphics{eda_files/figure-beamer/unnamed-chunk-6-1.pdf}

\end{frame}

\begin{frame}{Scatterplot Matrix for WISC\_4}
\protect\hypertarget{scatterplot-matrix-for-wisc_4}{}

\includegraphics{eda_files/figure-beamer/unnamed-chunk-7-1.pdf}

\end{frame}

\begin{frame}{WISC Total and Motor Skills}
\protect\hypertarget{wisc-total-and-motor-skills}{}

\begin{center}\includegraphics{eda_files/figure-beamer/unnamed-chunk-8-1} \end{center}

\end{frame}

\begin{frame}{WISC Total and Social Responsivness}
\protect\hypertarget{wisc-total-and-social-responsivness}{}

\begin{center}\includegraphics{eda_files/figure-beamer/unnamed-chunk-9-1} \end{center}

\end{frame}

\begin{frame}{WISC Total and Social Responsivness by Diagnosis}
\protect\hypertarget{wisc-total-and-social-responsivness-by-diagnosis}{}

\begin{center}\includegraphics{eda_files/figure-beamer/unnamed-chunk-10-1} \end{center}

\end{frame}

\begin{frame}{Conclusion}
\protect\hypertarget{conclusion}{}

In general, there appears to be a negative relationship between motor
skills and social responsiveness score. When broken down by primary
diagnosis, there are obvious differences in the relationship for each.
Specifically, in the ``None'' category, the relationship is close to
zero, while in ``Autism'' and ``ADHD'', the overall negative trend seems
to hold. Regarding our research question, when the data is not separated
by diagnosis, the relationship is hard to see. However, when broken down
by diagnosis, very noticable differences between the relationship
appear. Additionally, the relationships between WISC4 and SRS seem much
better defined than the WISC4 and mABC score, giving support that WISC
could add novel predictive ability.

\end{frame}

\end{document}
